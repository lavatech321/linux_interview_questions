\textbf{User and Group Administration}

\begin{enumerate}
    \item \textbf {What is a user?}
    \newline
    Ans. In Linux user is one who uses the system. 

	\bigskip
	\bigskip
    
    \item \textbf {How many types of users available in Linux?}
    \newline
    Ans.There are 5 types of users available in Linux.
    \begin{itemize}
	    \item System user   (Admin user who control the whole system nothing but root user).
	    \item Normal user  (Created by the Super user. In RHEL - 7 the user id's from 1000 - 60000).
	    \item System user   (Created when application or software installed 
	    \item In RHEL - 7 the System users are
	    Static system user id's from 1 - 200 and
	    (ii) Dynamic system user user id'sfrom 201 - 999).
	    \item Network user   (Nothing but remote user, ie., who are login to the system trough network  created
	    \item Windows Active Directory or in Linux LDAP or NIS). 
	    \item Sudo user   (The normal users who are having admin or Super user privileges)
    \end{itemize}

    
    \bigskip
    \bigskip
    
    \item \textbf { What is user management?}
    \newline
    Ans.User management means managing user. ie., Creating the users, deleting the users and modifying the users.
    
    \bigskip
    \bigskip
    
    \item \textbf{What are the important points related to users?}
    \begin {itemize}
	    \item Ans.Users and groups are used to control access to files and resources.
	    \item Users can login to the system by supplying username and passwords to the system.
	    \item Every file on the system is owned by a user and associated with a group.
	    \item Every process has an owner and group affiliation.
	    \item Every user in the system is assigned a unique user id (uid) and group id (gid).
	    \item User names and user id are stored in  /etc/passwd file.
	    \item User's passwords are stored in  /etc/shadow  file in an encrypted form.
	    \item Users are assigned a home directory and a shell to work with the O/S.
	    \item Users cannot read, write and execute each other's files without permission.
	    \item Whenever a user is created a mail box is created automatically in /var/spool/mail location.
	    \item And some user environmental files like  .bash\_logout,  .bash\_profile,  .bashrc ,  ...etc., are also copied from /etc/skell   to  his/her home directory (/home/<username>).
	\end{itemize}    
    
    \bigskip
    \bigskip

    \item \textbf{What are fields available in  /etc/passwd file?}
    \newline
    Ans.<user name>   :   x   :   <uid>   :   <gid>   :   <comment>   :
         <user's home directory>  :   <login shell
         (where   'x'   means link to password file ie.,  /etc/shadow   file)

    \bigskip
    \bigskip

    \item \textbf {What are fields available in  /etc/shadow  file?}
    \newline
    Ans. user name : password : last changed : min. days : max. days : warn days : inactive days : expiry days : reserved  for future.

    \bigskip
    \bigskip

    \item\textbf {What are the files that are related to user management?}
    \begin {itemize}
	    \item Ans.\textbf{/etc/passwd}: Stores user's information like user name, uid, home directory and shell ...etc.,
	    \item \textbf{/etc/shadow}: Stores user's password in encrypted form and other information.
	    \item \textbf{/etc/group}: Stores group's information like group name, gid and other information.
	    \item \textbf{/etc/gshadow}: Stores group's password in encrypted form.
	    \item \textbf{/etc/passwd}: Stores the  /etc/passwd   file backup copy.
	    \item \textbf{/etc/shadow}: Stores the /etc/shadow  file backup copy.
	    \item \textbf{/etc/default/useradd}: Whenever the user created user's default settings taken from this file.
	    \item \textbf{/etc/login.defs}: user's login defaults settings information taken from this file.
	    \item \textbf{/etc/skell}: Stores user's all environmental variables files and these are copied from this directory to user's home directory
    \end{itemize}

    \bigskip
    \bigskip

    \item \textbf {In how many ways can we create the users?}
    \begin{itemize}
	    \item Ans. \textbf{useradd} - <options><user name>
	    \item (ii)  \textbf{adduser}    - <options><user name>
	    \item (iii) \textbf{newusers}    <file name>   (In this file we have to enter the user details same as /etc/passwd   file)
    \end{itemize}

    \bigskip
    \bigskip

    \item\textbf{What is the syntax of useradd command with full options?}
    \newline
     Ans. useradd  -u <uid>  -g <gid>  -G <secondary group> -c  <comment>  -d <home 
        directory> -s <shell><user name> 
        \newline
        \textbf{Example:} useradd  -u  600  -g  600  -G  java   -c  "oracle user"   -d  /home/raju   -s /bin/bash raju

    \bigskip
    \bigskip

    \item\textbf {What is the syntax of adduser  command with full options?}
    \newline
     Ans.adduser  -u <uid>  -g <gid>  -G <secondary group> -c  <comment>  -d <home
    directory> -s <shell><user name>
    \newline
    \textbf{Example}\# adduser  -u  700  -g  700  -G  linux   -c  "oracle user"   -d  /home/ram   -s
    /bin/bash ram.

    \bigskip
    \bigskip
    
    \item\textbf{What is the syntax of newuser command?}
    \begin{itemize}
        \item Ans. newusers   <file name>	(This command will create multiple users at a time)
        \item First we should a file and enter user's data as fields same as the fields of /etc/passwd file for how many users do you want to create and mention that file as an argument for newusers command.
        \item When we execute this command new users will be created but their environmental files like        
    \end{itemize}  
        \textbf{.bash\_logout, .bash\_profile, .bashrc and .bash\_history}  files will not be copied from /etc/skell.
         Directory. So, we have to  copied manually from  /etc/skell   directory.

     \bigskip
     \bigskip
     
     \item\textbf{What is the syntax of userdel command with full options?}
     \newline
      Ans.userdel <options><user name>
      \newline
      The options are:
      \begin{itemize}
          \item \textbf{-f} :forcefully delete the user even through the user is login. The user's home directory, mail and message directories are also deleted. 
          \item \textbf{-r} : recursively means files in the user's home directory will be deleted and  his  home directory also deleted but the other files belongs to that user should be   deleted manually.
	  \end{itemize}
    \bigskip
    \bigskip

    \item\textbf{How to check whether user is already created or not?}
    \newline
     Ans.We can check in different ways:
     \begin{itemize}
     	\item id   <user name>   (It shows the user id group id and user name if that is already created)
     	\item grep <user name> /etc/passwd
     \end{itemize}
         
    \bigskip
    \bigskip
    
    \item\textbf{How to verify or check the integrity of the password file?}
    \newline
     Ans.\textbf{pwck}  <options>   /etc/passwd    or
    \newline
        \textbf{pwck} <options>   /etc/shadow 
        The options are,  \begin{itemize}
                            \item \textbf{-q} :quiet
                            \item \textbf{r }: read only	
                            \item \textbf{s }: sort the contents by uidin /etc/passwd and  /etc/shadow  files
                          \end{itemize}

    \bigskip
    \bigskip

    \item\textbf{How to verify or check the integrity of the group file?}
     \begin{itemize}
        \item grpck   <options>   /etc/group    or
        \item grpck   <options>   /etc/gshadow 
        \item The options are,    \textbf{-r}-r:read only
                                  \textbf{-s}:sort the contents by  \textbf{gidin  /etc/group   and   /etc/gshadow files.}
     \end{itemize}

    \bigskip  
    \bigskip
    
    \item\textbf{What is syntax of the usermod command with full options?}
    \newline
    Ans. usermod   <options><user name>
        The options are,        \begin{itemize}
                                  \item  -L:lock the password
                                  \item -U : unlock the password
                                  \item -o :creates duplicate user modify the user's  id  same as other user
                                  \item -u :modify user id
                                  \item -g : modify group id
                                  \item -G : modify   or  add  the  secondary group
                                  \item -c : modify comment
                                  \item -d : modify home directory
                                  \item -s : modify user's login shell
                                  \item -l : modify user's login name
                                  \item -md :modify the users home directory and the old home directory 
                                \end{itemize}
    \bigskip
    \bigskip

    \item\textbf{How to create the duplicate root user?}
    \newline
    Ans.useradd   -o   -u   0   -g   root    <user name>

    \bigskip
    \bigskip

    \item\textbf{How to recover if the user deleted by mistake?}
    \newline
    Ans.\textbf{pwunconv}	(It creates the users according  \textbf{/etc/passwd}  file and deletes the  \textbf{/etc/shadow   file)}

    \bigskip
    \bigskip
     
    \item\textbf{What are the uses of  .bash\_logout,.bash\_profile and .bashrc   files?}
    \newline
      \begin{enumerate}
        Ans.\item\textbf{.bash\_logout} :is a user's logout ending program file. It will execute first whenever the user is logout.
        \item\textbf{.bash_profile} :is user's login startup program file. It will execute first whenever the user is login. It consists 	 the user's environmental variables. 
        \item\textbf{bashrc} :This file is used to create the user's custom commands and to specify the umask values for that user's  only.
       \end{enumerate}

    \bigskip
    \bigskip

    \item\textbf{What is a group?}
    \newline
    Ans.The collection of users is called a group. There are two types of groups.
     \begin{enumerate}
          \item\textbf{Primary group} : It will be created automatically whenever the user is created. User belongs to on group is called a  primary group.
          \item\textbf{Secondary group} :  It will not create automatically. The admin user should be created manually and users belongs to more than one group is called secondary group. A user can be assigned to max. 16 groups. ie., 1 primary group and 15 secondary groups.
     \end{enumerate}

    \bigskip
    \bigskip

    \item\textbf{What is the command to check the user belongs to how many groups?}
    \newline
    Ans. groups     <user name>

     \bigskip
     \bigskip

    \item\textbf{What is the syntax to create the group?}
    \newline
    Ans.\textbf{groupadd}<options><group name>
    The options are,\begin{itemize}
                       \item -f :add the group forcefully
                       \item -g : group id no.
                       \item -o :non-unique  (duplicate group id)
                       \item -p : group password
                       \item -r : system group
                       \item -R : root group  
                    \end{itemize}

    \bigskip
    \bigskip
    
    \item\textbf{What is the syntax to modify the group?}
    \newline
    Ans.The options are,\begin{itemize}
                          \item -g : group id
                          \item-n :new name for existing one, ie., rename the group
                          \item-o : non-unique  (duplicate group id)
                          \item-p : group passwd
                          \item-R :root group
                        \end{itemize}

    \bigskip
    \bigskip

    \item\textbf{What is syntax to delete the group?}  
    \newline
      \begin{itemize}
        \item \textbf{groupdel}    <group name>	(to delete the group without options)
        \item \textbf{groupdel}   <group name>	(to delete the group without options)
       \end{itemize}
       
    \bigskip
    \bigskip
    
    \item\textbf{How to assign the password to the group?}
    \newline
    Ans.\textbf{gpasswd} <group name>	(to assign a password to the group without any options).
        \textbf{gpasswd}    <options><group name>
        The options are,            \begin{itemize}
             	                      \item -a :  add users to the group
				       	              \item -d :  delete the user from the group
					                  \item -r :  remove the group password
					                  \item -R :  restrict to access that group
					                  \item -A : set the list of Administrative users
					                  \item -M :  set the list of group members
                                    \end{itemize}

    \bigskip
    \bigskip

    \item\textbf{How to check the integrity or consistency of the group?}
    \newline
    Ans.grpck	(it will check the integrity or consistency in  \textbf{/etc/gpasswd }  and   /etc/gshadow   files).

    \bigskip
    \bigskip

    \item\textbf{How to restore  /etc/gshadow file if deleted by mistake?}
    \newline
    Ans. grpconv	(it creates the  \textbf{/etc/gshadow}  file  from   \textbf{/etc/group}   file)
    
    \bigskip
    \bigskip

    \item\textbf{How to change the password aging policies?}
    \newline
    Ans.we can change the password policies in 2 ways(I.e configuration file and Chage Command)
        (i)  First  open the  /etc/login.defs   file and modify the current values
        \newline
        \textbf{Example} :vim /etc/login.defs
           \begin{itemize}
             \item min - 0:means the user can change the password to any no. of times.
             \item min  -2: means the user can change the password within 2 days. ie., he can change the password after 2 days.
             \item max - 5:  means the user should change the password before or after 5 days. Otherwise the password will be expired after 5 days.
             \item inactive - 2 :means after password expiry date the grace period another 2 days will be given to change the password.
             \item warning - 7 : means a warning will be given to the user about the password expiry 7 days before expiry date. 
           \end{itemize}
        (ii)  second by executing the   # chage  command.
        \newline
        \textbf{Example} : chage   <options><user name>
        The options are     \begin{itemize}
                               \item -d : last day
                               \item -E : expiry date
                               \item -I : inactive days
                               \item -l :  list all the policies
                               \item -m :  min. days
                               \item-M  : max. days
                               \item-w  : warning days
                            \end{itemize}
         Note :Whenever we \textbf{change} the password aging policy using  chage   command, the information is will be modified in   /etc/shadow   file.

    \bigskip
    \bigskip

    \item\textbf{How add 45 days to the current system date?}
    \newline
    Ans. date   -d    "+  45 days"
   
    \bigskip
    \bigskip

    \item\textbf{Explain the sudo user?}
    \newline
    Ans.Sudoers (nothing but sudo users) allows particular users to run various root user commands without needing a root password. 
    \textbf{/etc/sudoers} is the configuration file for sudoers to configure the normal user as privileged user.
    It is not recommended to open this file using \textbf{vim }   editor because this editor cannot check the syntax by default and whatever we typed in that file that will blindly save in this file.
    So, one editor is specially available for opening this file, i.e.,\textbf{visudo}and all normal users cannot execute this command. Only root user can run this command.
    Once this file is opened nobody can open this file again on another terminal because \textbf{"The file is busy"}message is displayed on the terminal for security reasons.
    
    \bigskip
    \bigskip

    \item\textbf{How to give different  sudo permissions to normal users?}
    \newline
    Ans.Open the \textbf{/etc/sudoers} file by executing    #visudo    command and go to line no. 98 and type as 
    \textbf{<User name>	<Machine>=	<Command>root
           \newline
            ALL=(ALL)	        ALL
            \newline
	         raju		      All=		        ALL}Save and exit this file.
            \newline
    Note :  When we trying to save this file if any syntax errors in this file, those errors are displayed with line no's and \textbf{What you do ?} (will be displayed, here press  'e'  to edit this file and modify those errors or mistakes and save this file.
       \begin{itemize}
          \item su -  raju	(to switch to raju user)
          \item sudo   useradd   <useradd>	(The normal user raju can also add the users to the system)
          \item 	We can assign sudo permissions to  'n'  no. of users by specifying names separated by commas ( , ) or line by   line.
          \item	Instead of giving all permissions to normal user we can give only some commands.
          
        

    














     

                     


 


    


    


    



\end{enumerate}
