\item \textbf{User and Group Administration}
\begin{enumerate}
    \item \textbf {What is a user?}\newline
    Ans. In Linux user is one who uses the system. 
\bigskip
\bigskip
    \item \textbf {How many types of users available in Linux?}\newline
    Ans.There are 5 types of users available in Linux.
     \begin{itemize}
    \item System user   (Admin user who control the whole system nothing but root user).
    \item Normal user  (Created by the Super user. In RHEL - 7 the user id's from 1000 - 60000).
    \item System user   (Created when application or software installed 
    \item In RHEL - 7 the System users are\newline
    Static system user id's from 1 - 200 and \newline
    (ii) Dynamic system user user id'sfrom 201 - 999).\newline
    \item Network user   (Nothing but remote user, ie., who are login to the system trough network  created
    \item Windows Active Directory or in Linux LDAP or NIS). 
    \item Sudo user   (The normal users who are having admin or Super user privileges)
    \end{itemize}
    \item The types of users in Linux and their attributes:\newline
    table pending
    \item \textbf { What is user management?}\newline
    Ans.User management means managing user. ie., Creating the users, deleting the users and modifying the users.
    \bigskip
    \bigskip
    \item \textbf{What are the important points related to users?}
    \begin {itemize}
    \item Ans.Users and groups are used to control access to files and resources.\newline
    \item Users can login to the system by supplying username and passwords to the system.\newline
    \item Every file on the system is owned by a user and associated with a group.\newline
    \item Every process has an owner and group affiliation.\newline
    \item Every user in the system is assigned a unique user id (uid) and group id (gid).\newline
    \item User names and user id are stored in  /etc/passwd   file.\newline
    \item User's passwords are stored in  /etc/shadow  file in an encrypted form.\newline
    \item Users are assigned a home directory and a shell to work with the O/S.\newline
    \item Users cannot read, write and execute each other's files without permission.\newline
    \item Whenever a user is created a mail box is created automatically in /var/spool/mail  location.\newline
    \item And some user environmental files like  .bash_logout,  .bash_profile,  .bashrc ,  ...etc., are also copied from /etc/skell   to  his/her home directory (/home/<username>).\newline 
    \bigskip
    \bigskip
    \item \textbf{What are fields available in  /etc/passwd file?}\newline
          Ans.<user name>   :   x   :   <uid>   :   <gid>   :   <comment>   : \newline 
         <user's home directory>  :   <login shell\newline
         (where   'x'   means link to password file ie.,  /etc/shadow   file)\newline
    \bigskip
    \bigskip
    \item \textbf {What are fields available in  /etc/shadow  file?}\newline
          Ans. user name : password : last changed : min. days : max. days : warn days : inactive days : expiry days : reserved  for future.\newline
    \bigskip
    \bigskip
    \item\textbf {What are the files that are related to user management?}
    \begin {itemize}
    \item Ans./etc/passwd  -----> Stores user's information like user name, uid, home directory and shell ...etc.,\newline
    \item/etc/shadow  ----> Stores user's password in encrypted form and other information.\newline
    \item/etc/group  ------> Stores group's information like group name, gid and other information.\newline
    \item/etc/gshadow ---> Stores group's password in encrypted form.\newline
    \item/etc/passwd-  ---> Stores the  /etc/passwd   file backup copy.\newline
    \item/etc/shadow- ---> Stores the /etc/shadow  file backup copy.\newline
    \item/etc/default/useradd  ----> Whenever the user created user's default settings taken from this file.\newline
    \item/etc/login.defs  ----> user's login defaults settings information taken from this file.\newline
    \item/etc/skell   ------> Stores user's all environmental variables files and these are copied from this directory to user's home directory\newline
    \end{itemize}
    \bigskip
    \bigskip
    \item \textbf {In how many ways can we create the users?}
    \begin {itemize}
    \item Ans.(i)   # useradd    - <options><user name>
    \item (ii)  # adduser    - <options><user name>
    \item (iii) # newusers    <file name>   (In this file we have to enter the user details same as /etc/passwd   file)\newline
    \end {itemize}
    \bigskip
    \bigskip
    \item\textbf{What is the syntax of useradd command with full options?}
         Ans.useradd  -u <uid>  -g <gid>  -G <secondary group> -c  <comment>  -d <home \newline
        directory> -s <shell><user name>\newline
        \textbf {Example:}# useradd  -u  600  -g  600  -G  java   -c  "oracle user"   -d  /home/raju   -s \newline
     s  /bin/bash    raju\newline
    \bigskip
    \bigskip
    \item\textbf {What is the syntax of adduser  command with full options?}
           Ans.adduser  -u <uid>  -g <gid>  -G <secondary group> -c  <comment>  -d <home\newline
    directory> -s <shell><user name>\newline
    \textbf {Example}# adduser  -u  700  -g  700  -G  linux   -c  "oracle user"   -d  /home/ram   -s\newline  
    /bin/bash   ram.
